% 16/03/2018 Первая версия, разосланная участникам 
%            семинара Боголюбова. 
% 20/03/2018 Правки Севастьянова.
% 25/03/2018 Правки Боголюбова.
% 3/10/2018 Отослана Л.В. Раевской.
% ЖВМ и МФ. №2, 2019.
\documentclass[a4paper,12pt]{article}
\usepackage{misccorr}
\usepackage[centertags]{amsmath}
\usepackage{amsfonts}
\usepackage{amssymb}
\usepackage{amsthm}

%Параметры страницы 
\textheight 24 true cm
\textwidth 16.5 true cm
\hoffset -10 true mm
\voffset -25 true mm
\unitlength=1mm
\renewcommand{\baselinestretch}{1.25}
\sloppy
\frenchspacing
\usepackage{indentfirst}

%Окружения
\theoremstyle{remark}
\newtheorem*{Anm}{\rm \bf Замечание}
\newtheorem*{acknowledgements}{\rm \bf Благодарности}
\theoremstyle{definition}
\newtheorem{Def}{\rm \bf Определение}
\newtheorem{Ax}{\rm \bf Требование}
\newtheorem{Theorem}{\rm \bf Теорема}
\newtheorem{Folderung}{\rm \bf Следствие}
\newtheorem{Lemma}{\rm \bf Лемма}
\newtheorem{Ex}{\rm \bf Пример}
\newtheorem{Aufgabe}{\rm \bf Задача}
\providecommand{\keywords}[1]{\textbf{Ключевые слова:~} #1.}

%Мат. символы:
\def\bb{\mathbb}
\def\phi{\varphi}
\def\Phi{\varPhi}
\def\epsilon{\varepsilon}
\def\theta{\vartheta}
\def\rot{\operatorname{rot}}
\def\div{\operatorname{div}}
\def\HD{\overset{\mbox{\tiny 0}}{W}{}^1_2} 
\def\HN{{W}^1_2} 

%%% Local Variables:
%%% mode: latex
%%% coding: utf-8-unix
%%% End:
