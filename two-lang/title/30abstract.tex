\begin{Russian}
\begin{abstract}

% \begin{description}
% \textsc{Предпосылки.} 
%
%\textsc{Цель.} 
%
%\textsc{Методы.} 
%
%\textsc{Результаты.} 
%
%\textsc{Выводы.} 
% \end{description}

Автоколебательные режимы в системах управления в сетях передачи данных крайне
негативно влияют на характеристики этих сетей. Актуальной является
проблема нахождения зон возникновения автоколебаний, а также исследование
параметров автоколебаний. Из за существенной нелинейности
характеристик элементов управления изучение автоколебательных режимов
представляет определённые трудности. В работе описывается методика
исследования автоколебательных режимов на основе теории
управления. Данный материал носит скорее методический, чем
исследовательский характер.

\end{abstract}
\end{Russian}

\begin{English}
\begin{abstract}
Self-oscillating modes in control systems of computer networks  quite
negatively affect the characteristics of these networks. The problem of finding 
the areas of self-oscillations is actual and important as the study of
parameters of self-oscillations. Due to the significant nonlinearity of control
characteristics, the study of the self-oscillating modes
presents certain difficulties. This paper describes the  technique of research
of self-oscillating modes on the basis of the control theory. 
This material is rather methodical than exploratory one.
\end{abstract}
\end{English}




