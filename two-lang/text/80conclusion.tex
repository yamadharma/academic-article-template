\begin{Russian}
\section{Заключение}
\end{Russian}
\begin{English}
\section{Conclusion}
\end{English}
\label{sec:conclusion}

\begin{Russian}
Авторами была продемонстрирована методика исследования колебательных
режимов систем с управлением. Авторы постарались изложить эту методику
понятным образом математиков, незнакомых с формализмом теории
управления. В дальнейшем предполагается применить данную методику к
исследованию широкого спектра алгоритмов активного управления
трафиком. Также представляется интересным сравнить эти результаты с ранее
полученными результатами по областям возникновения автоколебаний в
системах с управлением.
\end{Russian}
\begin{English}
The authors demonstrated the technique of research of oscillatory
modes of the systems with control. We tried to explain this technique
for mathematicians unfamiliar with the formalism of the control theory. 
We plan to apply this technique to
the study of a wide range of algorithms of traffic active control. 
Also it is interesting to compare these results with the previous
results obtained for self-oscillation systems with control.
\end{English}


%%% Local Variables:
%%% mode: latex
%%% coding: utf-8-unix
%%% End:
